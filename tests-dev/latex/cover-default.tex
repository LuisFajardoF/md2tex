\documentclass{article}

\usepackage{graphicx}

\begin{document}
	\begin{figure}[!t]
		\centering
		\includegraphics[width=0.2\textwidth]{../images/mylogo.png}
	\end{figure}
	\title{Leyes de la termodin\'amica}
	\author{Nicolas Sadi Carnot \and Rudolf Clausius \and William Thompson}
	\date{15 de marzo del 1860}
	\maketitle
	Las leyes de la termodin\'amica (o los principios de la termodin\'amica) describen el 
	comportamiento de tres cantidades f\'isicas fundamentales, la temperatura, la energ\'ia 
	y la entrop\'ia, que caracterizan a los sistemas termodin\'amicos. El t\'ermino 
	\textbf{termodin\'amica} proviene del griego \textit{thermos}, que significa \textbf{calor}, y \textit{dynamos}, 
	que significa \textbf{fuerza}.\newline
  
	Matem\'aticamente, estos principios se describen mediante un conjunto de ecuaciones que 
	explican el comportamiento de los sistemas termodin\'amicos, definidos como cualquier objeto 
	de estudio (desde una mol\'ecula o un ser humano, hasta la atm\'osfera o el agua hirviendo en 
	una cacerola).\newline
  
	Existen cuatro leyes de la termodin\'amica y son cruciales para comprender las leyes f\'isicas 
	del universo y la imposibilidad de ciertos fen\'omenos como el del movimiento perpetuo.
\end{document}