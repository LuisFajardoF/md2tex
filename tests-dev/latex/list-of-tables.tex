\documentclass{article}

\usepackage{siunitx}
\usepackage{booktabs}
\usepackage{pgfplotstable}
\usepackage{xcolor}
\usepackage{colortbl}

\sisetup{
	round-mode = places,
	round-precision = 2
}

\begin{document}
	\listoftables

	\begin{table}[h!]
		\centering
		\begin{tabular}{lSS} 
			\toprule
			\textbf{    Value 1 } & \textbf{Value 2  } & \textbf{Value 3} \\ \midrule
			    1       & 1110.10 & 0.00001 \\ 
			    2       & 10.10   & 0.9090 \\ 
			    3       & 23.11   & 0.8900 \\ 
			\bottomrule
		\end{tabular}
		\caption{Tabla con unidades alineadas en dos columnas.}
	\end{table}

	\begin{table}[h!]
		\centering
		\pgfplotstabletypeset[
			col sep=comma,
			string type,
			assign column name/.style={/pgfplots/table/column name={\textbf{#1}}},
			every head row/.style={before row=\toprule, after row=\midrule},
			every last row/.style={after row=\midrule},
		]{../csv/table1.csv}
		\caption{Valores aleatorios para la Tabla 1.}
	\end{table}

	\begin{table}[h!]
		\centering
		\pgfplotstabletypeset[
			col sep=comma,
			string type,
			every head row/.style={before row=\toprule, after row=\midrule},
			assign column name/.style={/pgfplots/table/column name={\textbf{#1}}},
			every even row/.style={after row ={\rowcolor{lime!30}}, before row={\rowcolor{lime!10}}},
			every last row/.style={after row=\midrule},
		]{../csv/table2.csv}
		\caption{Valores aleatorios para la Tabla 2.}
	\end{table}
\end{document}